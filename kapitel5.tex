%
%	Fazit
%

\pagebreak
\section{Summary and recommendations}

\onehalfspacing

As recommended by NIST, most Enterprise IT will most likely end up with more than one Kubernetes cluster in production, either permanently as part of the infrastructure or ephemeral as part of the application deployment.

In this paper we have seen that Rancher currently provides an excellent tool to manage multiple Kubernetes clusters in Enterprise IT and offers all the necessary tools to operate Kubernetes in production; Rancher also provides mechanisms for governance, security, and compliance.

There are other options though: Google are developing Anthos, AWS have Outpost and Microsoft have Azure Arc (in preview) - all these tools are aimed at extending the management capabilities of the respective cloud provider to other public cloud platforms and in-house data centers, and provide capabilities to manage multiple clusters.

For the time being, Rancher is the only open-source, provider-independent solution and the recommended choice though.

There are new changes on the horizon that threaten infrastructure itself: With the advent of functions and Event-Driven Architecture, new patterns emerge in cloud native application design that have the potential to render virtualization and containerization obsolete. The main contenders in this space are AWS Lambda, Google Cloud Run, Google Cloud Functions, and Microsoft Functions.

Also, AWS and Cloudflare are developing new container run-time environments and micro-VMs; the biggest announcement at the end of 2019 was the introduction of Kubernetes on Fargate by AWS\footnote{Vgl. \textit{AWS (2019)}: Run Serverless Kubernetes Pods Using Amazon EKS \cite{eksFargate}}.

There are a lot of developments coming in 2020 with uncertain outcomes, however, what we can say for sure is that software will eat the world\footnote{Vgl. \textit{Andreessen, M. (2011)}: Why Software Is Eating the World \cite{softwareEats}}.
