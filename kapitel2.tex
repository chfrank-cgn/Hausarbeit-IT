%
%	Begrifflichkeiten
%

\pagebreak
\section{Kubernetes Terms and Concepts}

\onehalfspacing

\subsection{Kubernetes and Cloud Computing}

What exactly is Kubernetes? "Kubernetes (K8s) is an open-source system for automating deployment, scaling, and management of containerized applications.
It groups containers that make up an application into logical units for easy management and discovery. Kubernetes builds upon 15 years of experience of running production workloads at Google, combined with best-of-breed ideas and practices from the community."\footnote{\textit{The Linux Foundation (2019)}: Production-Grade Container Orchestration. \cite{kubernetes}}

Also, according to the National Institute of Standards and Technology (NIST), the key characteristics of cloud computing are: 
\begin{itemize}
\item On-demand self-service
\item Broad network access
\item Resource pooling
\item Rapid elasticity
\item Measured service\footnote{See \textit{Mell, P. (2011)}: The NIST Definition of Cloud Computing. \cite{sp800-145}}
\end{itemize}

Gartner similarly defines cloud computing, as "Cloud computing is a style of computing in which scalable and elastic IT-enabled capabilities are delivered as a service using Internet technologies."\footnote{\textit{Gartner (2019)}: Gartner Glossary - Cloud Computing. \cite{gartnerGlossary}}

Cloud computing is also the foundation for Continuous Integration / Continuous Deployment (CI/CD) in agile software development.

Sitting on top of the Docker container run-time, Kubernetes is the ideal tool to enable cloud computing and CI/CD on Public Cloud as well as for in-house data centers. It provides a rich set of API calls, aimed at application development and deployment, and is an easy tool to use for automated software development.

Kubernetes is being widely adopted, not only in the commercial space but also in the military. "It’s a flexible but universal development platform for software teams across the military and prevents vendor-lock in",\footnote{\textit{Chaillan, N. (2019)}: How the U.S. Air Force Deployed Kubernetes. \cite{airForce}} Nicolas Chaillan explained during his presentation on why the US Air Force chose Kubernetes for their F-16 fighter jets.

\subsection{Kubernetes Terms and Concepts}

The key terms and concepts in Kubernetes are:
\begin{itemize}
\item Pod
\item Replica Set
\item Service
\item Deployment
\end{itemize}

A Pod in Kubernetes is the smallest unit that can be deployed (ephemeral primitive). A Pod consists of one or more containers that are always scheduled together on the same node. Each pod is given a unique IP address and containers in a pod can speak to each other via localhost

A Replica Set defines the desired scale and state of a group of pods. Replica sets are not used directly, however, the resource needs to be understood as it is the based building block for building applications on Kubernetes. Replica sets can (when instructed to) scale up or down the number of pods that are desired.

A Services Defines a DNS name that can be used to refer to a group of pods; the service thus provides a consistent endpoint for that group of pods and aids service discovery. There are different types of services (nodePort, clusterIP, and load balancer), a more detailed explanation would exceed the scope of this text.

A Deployment is the level of abstraction above replica sets; deployments create and update replica sets and allow to easily scale applications and perform rolling upgrades if needed.

\subsection{Kubernetes Architecture}

Generally speaking, a Kubernetes cluster consists of one or more systems, grouped together. These systems can be either physical nodes, or better, be virtualized.

A cluster consists of at least one Master node, the so-called Control Plane, and one or more worker nodes, the Execution Plane. Most commonly, three nodes are being used for the control plane, and three or more nodes for the execution plane, sometimes with different sizing or capabilities.

The control plane takes care of orchestrating the application deployments and maintains their state, the worker nodes execute the actual application pods as defined and scheduled by the control plane.

All access to a Kubernetes cluster is through the master nodes and the Kubernetes API endpoints it provides. Kubernetes has a command-line interface tool, kubectl, to access the API; a detailed description can be found in the Kubernetes online documentation.\footnote{See \textit{The Linux Foundation (2019)}: Overview of kubectl. \cite{kubectl}} Besides the command line interface, most development pipelines now come with native Kubernetes integration.
