%
%	Begrifflichkeiten
%

\pagebreak
\section{Kubernetes Terms and Concepts}

\onehalfspacing

\subsection{Cloud Computing}
According to the National Institute of Standards and Technology (NIST), the key characteristics of cloud computing are 
\begin{itemize}
\item On-demand self-service
\item Broad network access
\item Resource pooling
\item Rapid elasticity
\item Measured service\footnote{Vgl. \textit{Mell, Peter (2011)}: The NIST Definition of Cloud Computing \cite{sp800-145}}
\end{itemize}

Gartner defines Cloud Computing in a similar way, as "Cloud computing is a style of computing in which scalable and elastic IT-enabled capabilities are delivered as a service using internet technologies."\footnote{\textit{Gartner (2019)}: Gartner Glossary - Cloud Computing \cite{gartnerGlossary}}

Sitting on top of the Docker\footnote{Vgl. \textit{Docker (2019)}: Enterprise Container Platform \cite{docker}} container run-time, Kubernetes\footnote{Vgl. \textit{The Linux Foundation (2019)}: Production-Grade Container Orchestration \cite{kubernetes}} enables Cloud Computing on Public Cloud as well as for in-house data centers.

\subsection{Kubernetes Concepts}

The key concepts of Kubernetes are:
\begin{itemize}
\item Pod
\item Replica Set
\item Deployment
\end{itemize}

Kubernetes consists of a cluster of systems, separated into Master nodes (Control Plane) and Worker nodes (Execution Plane)
