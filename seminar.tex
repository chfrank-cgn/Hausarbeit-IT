% Angepasste Seminarvorlage
%
\documentclass[12pt,a4paper]{scrartcl}

\usepackage[english]{babel}			% englische Namen/Umlaute
\usepackage[utf8]{inputenc}	    	% Zeichensatzkodierung
\usepackage{fancyhdr}
\usepackage{graphicx}				% Einbinden von Bildern
\usepackage{hyperref}				% Klickbare Verweise und \autoref{label}
\usepackage{nomencl}
\renewcommand{\familydefault}{\sfdefault}
\usepackage{geometry}
 \geometry{
 a4paper,
 left=40mm,
 right=20mm,
 top=40mm,
 bottom=20mm,
 }
\usepackage{setspace}
 
%
%	Hier werden Titel, Bearbeiter und das Datum eingetragen
%
\newcommand\svthema{Multi-Cluster Management für Containerumgebungen}
\newcommand\svperson{Christian Frank (\#473088)}
\newcommand\svdatum{\today} % \today für das aktuelle Datum, sonst als Text eintragen
\newcommand\lvname{Wirtschaftsinformatik: IT-Infrastruktur}
\newcommand\lvtyp{WS 2019}
\newcommand\lvinst{FOM - Hochschule für Oekonomie \& Management}
\newcommand\lvbetr{Dipl.-Wirtschaftsinf. (FH) Frank Rudolf Becker}

% Das ist ein 100 Woerter umfassender Blindtext, um das Aussehen von längeren Texten zu testen.
\newcommand{\blindtext}{Lorem ipsum dolor sit amet, consetetur sadipscing elitr, sed diam nonumy eirmod tempor invidunt ut labore et dolore magna aliquyam erat, sed diam voluptua. At vero eos et accusam et justo duo dolores et ea rebum. Stet clita kasd gubergren, no sea takimata sanctus est Lorem ipsum dolor sit amet. Lorem ipsum dolor sit amet, consetetur sadipscing elitr, sed diam nonumy eirmod tempor invidunt ut labore et dolore magna aliquyam erat, sed diam voluptua. At vero eos et accusam et justo duo dolores et ea rebum. Stet clita kasd gubergren, no sea takimata sanctus est Lorem ipsum dolor sit amet.}

\hypersetup{ % Thema und Author in die Meta-Daten der PDF
pdftitle={\svthema}, 
pdfauthor={\svperson}
}	
	
\begin{document}

% Hier wird der Titel-Bereich formatiert
% Die zuvor definierten Textbausteine werden hier verwendet.
\title{ \huge\textbf{\svthema} }
\author{ {\svperson} \\ \svdatum }
\date{ \normalsize \centering \includegraphics[width=0.3\textwidth]{FOM}\\ {\lvname} \\ {\lvbetr} \\ {\lvinst} \\ {\lvtyp} }
% eigene Veraenderung- FOM Logo mit eingefügt auf dem Deckblatt

% Seitennummer oben
\pagestyle{fancy}
\fancyhf{}
\fancyhf[ch]{\thepage}
\renewcommand\headrulewidth{0pt}

\maketitle
\thispagestyle{empty} % laesst die Seitennummer auf der Titelseite verschwinden
\pagenumbering{Roman}

\begin{abstract}
In this paper we'll first have a look at container technologies and orchestration frameworks. After introducing the most popular orchestration framework, Kubernetes, we'll look at challenges for Enterprises to manage multiple Kubernetes clusters.
For a possible solution we'll look at Rancher from Rancher Inc., a solution that offers very interesting features to manage multiple clusters. In closing we'll look at coming developments in container deployments.

\end{abstract}

\cleardoublepage

\tableofcontents			% Inhaltsverzeichnis ... nicht für kurze Dokumente!
\cleardoublepage

\listoffigures				% Abbildungsverzeichnis ... nicht für kurze Dokumente
\cleardoublepage

%
% Abkuerzungsverzeichnis
%
\makenomenclature
\renewcommand{\nomname}{List of Abbreviations}

\nomenclature{\textbf{IaaS}}{Infrastructure as a Service}
\nomenclature{\textbf{IaC}}{Infrastructure as Code}
\nomenclature{\textbf{K8s}}{Kubernetes}
\nomenclature{\textbf{NIST}}{National Institute of Standards and Technology}
\nomenclature{\textbf{PaaS}}{Platform as a Service}
\nomenclature{\textbf{SaaS}}{Software as a Service}

\printnomenclature
\cleardoublepage

\pagenumbering{arabic}
\setcounter{page}{4}

%
%	Einfuehrung
%

\pagebreak
\section{Introduction to Cloud Computing and Container Orchestration Frameworks}

\onehalfspacing

\subsection{Enterprise IT}

Ever since the start of the decade, Enterprise IT has undergone a massive transformation towards a utility-like service business. As of 2019, in Enterprise IT, Cloud Computing is now the new norm according to Gartner Research,\footnote{See \textit{Gartner (2019)}: Cloud computing is the new norm. \cite{gartnerCloudStatement}} and is expected to grow even further.\footnote{See \textit{Gartner (2019)}: Gartner Forecasts Worldwide Public Cloud Revenue. \cite{gartnerForecast}}

In the early days, compute transformation was focused on virtualization, whereas cloud computing now focuses on containerization. Both technologies are quite old, virtualization started in the early 70s, pioneered by IBM, with VM/CMS; the first attempt at containerization was made with the implementation of chroot() for Unix System V in the late 70s.

Outside of the mainframe world virtualization technologies were not widespread until VMware commoditized virtualization with ESX/ESXi in the early 00s. Around virtualization, a new ecosystem of self-service portals appeared, such as Microfocus' Cloud Service Automation. Enterprise virtualization with a self-service portal is not cloud computing though - for cloud computing to come to life, the first real open-source cloud operation system, OpenStack was needed, together with the arrival of the major public cloud providers (Amazon Web Services, Microsoft Azure, Google Cloud, Alibaba Cloud).

AWS started initially by providing excess compute capacity to its customers from its internal platforms, before turning into one of the biggest IT providers worldwide.

With all this raw compute power now available on tap, there is no real reason for Enterprise IT anymore to operate in-house data centers.

\subsection{Container Run-time}

During that time, Docker pioneered the first simple orchestration environment to run containers,\footnote{See \textit{Docker (2019)}: Enterprise Container Platform. \cite{docker}} on a single node. Whereas virtualization is focused on virtual compute instances, containerization is focused on application delivery and deployment; with immutable images and a focus on automation, it is fully aimed at the development of cloud-native applications.\footnote{See \textit{Wiggins, A. (2017)}: The Twelve-Factor App. \cite{12factor}}

It is crucial to understand the difference: Whereas virtualization was aimed at the infrastructure level, containerization is aimed at (agile) application development and deployment.

Docker evolved and spawned other container run-timer environments, such as containerd and podman, but they mostly remained focused on executing on a single node. To orchestrate containers on more than one node, an orchestration framework is needed. 

Here are the most popular in 2019:
\begin{itemize}
\item Docker Swarm
\item Kubernetes
\item Mesos DC/OS
\end{itemize}

After Mirantis acquired the Docker Enterprise business, it announced the end of life for Docker Swarm in 2021\footnote{See \textit{Mirantis (2019)}: What We Announced Today and Why it Matters. \cite{mirantisDocker}}; Mesos DC/OS never gathered a large following, so as of the time of writing, CNCF's Kubernetes remains as the only container orchestration framework with a sizable installed base and that is under active development.

In this paper, we'll have a look at one of the challenges posed by introducing Kubernetes as a container orchestration framework into Enterprise IT and showcase a possible solution.


%
%	z.B. Text des zweiten Autors
%

\section{Weitere Kommandos}

\subsection{Mathematische Formeln}

Mathematische Formeln werden mittels \verb|\(...\)|
in den Fließtext eingebaut
--- zum Beispiel \( E=mc^2 \) und:
sei \(V\) ein Vektorraum über \(\mathbb{R}\)
und \(\mathcal{M}\) eine Indexmenge
---
oder aber mittels \verb|\[...\]|
abgesetzt und zentriert dargestellt:
	\[
	\pmb{x} = \sqrt[3]{\frac{a^2-b^2}{a^2+b^2}}
		~~~~~ \text{versus} ~~~~~
	\boldsymbol{x} = \sqrt[3]{\frac{a^2-b^2}{a^2+b^2}}
	\]


\subsection{Silbentrennung}
Vertrauen Sie bitte nie einer automatischen Silbentrennung (auch nicht der von Microsoft Word \& Co.). In folgendem Test-Absatz ist das Wort "`Spracherkennung"' falsch getrennt.

Testzeile Testzeile Testzeile Testzeile Testzeile Testzeile Testzeile Testzeile Te Spracherkennung.

Sie können LaTeX die richtige Trennung mit \verb|\hyphenation{...}| mitteilen. Man tut das üblicherweise noch vor \verb|\begin{document}|.

\hyphenation{Sprach-er-ken-nung}	% hier jetzt zu Demonstrationszwecken im Text
Testzeile Testzeile Testzeile Testzeile Testzeile Testzeile Testzeile Testzeile Te Spracherkennung.


\subsection{Literaturzitate}
\label{sec:literaturzitate}

Im Lehrbuch \cite{Schukat-Talamazzini1995}
finden sich Hinweise auf einschlägige Verfahren der automatischen Spracherkennung.

Das Literaturverwaltungsprogramm JabRef \cite{Kopp2018} ist für viele Plattformen verfügbar und unterstützt bei der Literaturrecherche. Es ist prädestiniert dazu, mit {\LaTeX} in Kombination mit {Bib\TeX} zusammenzuarbeiten.

Über die Literaturrecherche haben Sie Zugriff auf das Buch "`Das Textverarbeitungssystem LaTeX"' \cite{Oechsner2015}. Hierzu können Sie auch direkt den DOI-Link im Literaturverzeichnis anklicken.

\subsection{Einbinden von Bildern}

Sie können mit der Anweisung \verb|\includegraphics{datei}| eine Grafikdatei einbinden. Diese kann im PDF-, JPG- oder PNG-Format vorliegen. Grafiken werden üblicherweise in die float-Umgebung \verb|figure| gekapselt. Der Assistent in TeXstudio \cite{vanderZander2018} tut dies automatisch, wenn das Verhalten nicht explizit abgeschaltet wird. Jede eingebundene Grafik muss vom Text aus referenziert werden. Die textuelle Referenz hat \textit{vor} der Grafik zu erfolgen. Ein eingebundenes PDF ist in \autoref{fig:AufbauMustererkennungssystem} zu sehen.

\begin{figure}[tb]
\centering
\includegraphics [height=20mm] {images/bildchen}
\caption {Aufbau eines Mustererkennungssystems}
\label{fig:AufbauMustererkennungssystem}
\end{figure}

Auch andere Grafikformate werden unterstützt. Verwenden Sie den Assistenten in TeXstudio, um komfortabel Grafiken einzufügen. Siehe dazu \autoref{fig:GrafikEinfuegen}. Die Grafiken finden sich nicht notwendigerweise direkt am Einfüge-Ort. Das Textsatzsystem richtet es so ein, dass es gut aussieht.

\begin{figure}
\centering
\includegraphics[width=0.7\linewidth]{images/GrafikEinfuegen}
\caption{Verwendung des Assistenten, um eine Grafik einzufügen.}
\label{fig:GrafikEinfuegen}
\end{figure}

\blindtext

\blindtext

\blindtext
\blindtext

\subsection{Querverweise}\label{sec:querverweise}
Verwenden Sie unter TeXstudio die rechte Maustaste in der Strukturübersicht, um für einen Abschnitt ein Label zu erzeugen, auf das Sie Bezug nehmen können (siehe \autoref{fig:LabelErzeugen}).

\begin{figure}
\centering
\includegraphics[width=0.4\linewidth]{images/LabelErzeugen}
\caption{Automatische Erzeugung eines Labels für Querverweise}
\label{fig:LabelErzeugen}
\end{figure}

Es wird ein Eintrag \verb|\label{key}| erzeugt, auf den man beispielsweise mit
\verb|\autoref{key}| verweisen kann. Verwendet man \verb|\autoref|, wird der Typ des Objekts
(z.\,B. Abbildung, Tabelle, etc.) mit ausgegeben. Verwendet man nur \verb|\ref|,
so wird nur die Nummerierung des Objekts ausgegeben.

\subsection{Erstellen von Tabellen}

Das Volk hat gesprochen. Siehe \autoref{tab:tabular}. Auch hier kommt ein float zum Einsatz, jedoch mit dem Positionierungs-Hinweis \verb|[h]|. Jedes float sollte mit einem Label versehen werden, und es sollte im Text darauf verwiesen werden, da sich die Position ändern kann.

\begin{table}[h]
\centering
\begin{tabular} {|llc||r|}
	\hline
	Name & Rang & Fraktion & Stimmenanteil \\
	\hline
	Mobutu & General & CDU & 57\% \\
	Tsvangirai & Oberst & CSU & 63\% \\
	\hline
\end{tabular}
\caption {Bundestagswahl in Simbabwe}
\label{tab:tabular}
\end{table}

\autoref{tab:booktab} verwendet keine vertikalen Linien und entspricht dem üblicherweise in Büchern verwenden Stil. Derartige Tabellen sind deutlich ansehnlicher.

\begin{table}[h]
\centering
\begin{tabular} {llcr}
	\toprule
	Name & Rang & Fraktion & Stimmenanteil \\
	\midrule
	Mobutu & General & CDU & 57\% \\
	Tsvangirai & Oberst & CSU & 63\% \\
	\bottomrule
\end{tabular}
\caption {Bundestagswahl in Simbabwe}
\label{tab:booktab}
\end{table}

\subsection{Listen und Aufzählungen}

Listen und Aufzählungen werden in einer Umgebung angelegt (umschlossen von \verb|begin| und \verb|end|.). \verb|\begin{itemize}| leitet eine Liste ein und \verb|\begin{enumerate}| eine Aufzählung. Die Einträge werden jeweils mit \verb|\item| begonnen. Folgend zwei Beispiele.

Itemize:
\begin{itemize}
\item Test
\item Test
\item Test
\end{itemize}

Enumerate:
\begin{enumerate}
\item Test
\item Test
\item Test
\end{enumerate}

%
%	Theorieteil
%

\pagebreak
\section{Management of Container Platforms in the Enterprise}

\onehalfspacing

\subsection{Key requirements for Kubernetes cluster in Enterprise IT}

Once the transition to cloud-native application development begins and containers are introduced to IT production, Day Two operations and security become the main concerns.

One of the crucial components of security when running containers in production, according to NIST, is the separation of applications and systems.\footnote{See \textit{Souppaya, M. (2017)}: Application Container Security Guide. \cite{sp800-190}}

Segmentation of applications could be performed along the lines of function (Production, Development/Test), or between applications, or both. In single-cluster environments, separation could be achieved through networking or else through the introduction of multiple clusters. 

\begin{figure}[h]
\centering
\caption {Cluster Separation}
\includegraphics[width=\linewidth]{images/separation}
\label{fig:clusterSeparation}
\end{figure}

A key consideration when implementing separation is the so-called "Blast Radius", a term borrowed from the military, which depicts the amount of damage an explosive would cause. In IT, it is used to assess the damage a breach, data loss, or failure of a given IT system would cause to the whole operation, in technical and financial terms.

Defining application and data security classes is part of the preparation process when introducing containers to production, and would exceed the scope of this paper.

Kubernetes itself, at the time of writing, does not provide for hard tenancy. To fully separate applications on all layers (compute, network, and storage), the use of multiple clusters is a good option. Many enterprises might thus end up with more than one Kubernetes cluster, sometimes with many more, which will, in turn, have a great effect on operations.

\subsection{The principle of least privilege}

The second crucial component of security is user authentication and authorization.

Many Enterprise IT already have some form of central user authentication, through Microsoft Active Directory or other Single-Sign-On providers. Kubernetes has Role-Based Access Control and can restrict access to its resources based on the roles bound to a specific user or object.

Defining Roles and Responsibilities is part of the Role Engineering process during the introduction of containers, and would exceed the scope of this paper. It is, however, important in RBAC to fully embrace the principle of least privilege - only ever grant the minimum access rights required to fulfill the job, without jeopardizing efficiency though. 

A comprehensive reference for RBAC can be found in David F. Ferraiolo's 2007 book of the same name, Role-Based Access Control\cite{rbac}.

\subsection{Organizational considerations}

The third crucial component of security is departmental organization - NIST recommends that an organization changes its operational culture and technical processes to fully embrace and support agile application development.

Agile software development requires an agile organizational structure and self-organized teams to support the agile mantra “You build it, you run it” (Werner Vogels).\footnote{\textit{Orban, S. (2015)}: Enterprise DevOps: Why You Should Run What You Build. \cite{devOps}}

Without such changes, from experience, the likelihood of failing the transformation to container-based, cloud-native application development is quite high.

\subsection{Day Two operations with multiple Kubernetai}

Once all preparations have been completed, the organizational structure is about to change and multiple clusters are installed, there are a lot of configuration items and actions that need to be synchronized across all clusters:

\begin{itemize}
\item User Authentication
\item Roles and Responsibilities
\item Security Policies
    \begin{itemize}
    \item Pod Security Policies
    \item Network Security Policies
    \end{itemize}
\item Version control and upgrades
    \begin{itemize}
    \item Kubernetes
    \item Applications
    \end{itemize}
\item Logging and Monitoring
\item Backup and Restore
\item Persistent volumes and storage classes
\end{itemize}

For software development and automatic deployment, it makes sense to connect the development pipeline(s) to the respective clusters and establish central governance for the deployment workflows.

For micro-service observability, distributed tracing and network control, a service mesh could be installed - a quite popular choice would be Istio.\footnote{See \textit{Istio Authors (2019)}: Istio - Connect, secure, control, and observe services. \cite{istio}}

All of these tasks can be performed against individual clusters with native Kubernetes tools, but once there are a couple of clusters to manage, this will get tedious and quite difficult to synchronize.

It will even get more difficult, once IaC enters the picture and Kubernetes clusters are starting to become ephemeral and the platform recreated fresh with each new application deployment, relying on automation to implement governance and policies.

\subsection{Possible solutions}

To address all these issues, CNCF is working on federation and multi-cluster controlling, but the development still in its infancy and not production-ready yet.\footnote{See \textit{sig-multicluster (2019)}: Kubernetes Cluster Federation. \cite{kubeFed}}

The three big cloud providers also provide solutions to manage multiple clusters: Google Cloud has Anthos,\footnote{See \textit{Google (2019)}: Anthos - Bringing the cloud to you. \cite{googleAnthos}} Microsoft Azure has Azure Arc in Preview,\footnote{See \textit{Microsoft (2019)}: Bring Azure services and management to any infrastructure. \cite{azureArc}} and AWS is working on AWS Outposts.\footnote{See \textit{AWS (2019)}: AWS Outposts. \cite{awsOutposts}}

A popular open-source solution for this problem is Rancher, by Rancher Labs.\footnote{See \textit{Rancher Labs (2019)}: Run Kubernetes Everywhere. \cite{rancher}} In the next chapter, we'll look at whether Rancher could provide a solution to multi-cluster management and become the tool of choice in Enterprise IT.


%
%	Praxisbezug
%

\pagebreak
\section{Using Rancher as an Enterprise Container Management Platform}

\onehalfspacing

\subsection{Rancher overview}

Rancher provides a solution to manage multiple Kubernetes clusters in Enterprise IT, all from an user-friendly GUI

Rancher was recognized in 2019 as a Firestarter\footnote{Vgl. \textit{Rancher Labs (2019)}: Rancher Labs Recognized by 451 Research as a ‘451 Firestarter’ \cite{firestarter451}} in Gartner's annual technology award program.

The Rancher Dashboard for an active Kubernetes cluster looks like this:

\begin{figure}[h]
\centering
\includegraphics[width=\linewidth]{images/dashboard}
\caption {Rancher Dashboard}
\label{fig:rancherDashboard}
\end{figure}

This cluster was created on Azure, note the enabled monitoring 

\subsection{Cluster Management}

Central cluster management and installation

Central version control and upgrades

Central Logging and Monitoring

Automatic backup 

Cluster, Node and Credential templates

\subsection{User Management}

Central user authentication

Central RBAC management

\subsection{Security}

Central Policy management

Central cluster hardening\footnote{Vgl. \textit{Rancher Labs (2019)}: Hardening Guide - Rancher v2.3.x \cite{hardeningGuide}}

Central secret management

\subsection{Applications and CI/CD}

Integrated Service Mesh

Integrated pipeline

Integration with Helm

Application catalogs

\subsection{Infrastructure as Code}

Terraform\footnote{Vgl. \textit{HashiCorp (2019)}: Deliver infrastructure as code with Terraform \cite{terraform}} provider\footnote{Vgl. \textit{Rancher Labs (2019)}: Introducing the Rancher 2 Terraform Provider \cite{terraformProvider}}


%
%	Fazit
%

\pagebreak
\section{Summary and recommendations}

\onehalfspacing

As recommended by NIST, most Enterprise IT will most likely end up with more than one Kubernetes cluster in production, either permanently as part of the infrastructure or ephemeral as part of the application deployment.

In this paper we have seen that Rancher currently provides an excellent tool to manage multiple Kubernetes clusters in Enterprise IT and offers all the necessary tools to operate Kubernetes in production; Rancher also provides mechanisms for governance, security and compliance.

There are other options though: Google are developing Anthos, AWS have Outpost and Microsoft have Azure Arc (in preview) - all these tools are aimed at extending the management capabilities of the respective cloud provider to other public cloud platforms and in-house data centers, and provide capabilities to manage multiple clusters.

For the time being, Rancher is the only open-source, provider independent solution and the recommended choice though.

There are new changes on the horizon that threaten infrastructure itself: With the advent of functions and Event Driven Architecture, new patterns emerge in cloud native application design that have the potential to render virtualization and containerization obsolete. Main contenders in this space are AWS Lambda, Google Cloud Run, Google Cloud Functions, and Microsoft Functions.

Also, AWS and CloudFlare are developing new container run-time environments and micro-VMs; the biggest announcement at the end of 2019 was the introduction of Kubernetes on Fargate by AWS\footnote{Vgl. \textit{AWS (2019)}: Run Serverless Kubernetes Pods Using Amazon EKS \cite{eksFargate}}.

There are a lot of developments coming in 2020 with uncertain outcomes, however, what we can say for sure is that software will eat the world\footnote{Vgl. \textit{Andreessen, M. (2011)}: Why Software Is Eating the World \cite{softwareEats}}.


% Das Literaturverzeichnis
\cleardoublepage
\bibliographystyle{IEEEtran}	% ieeetran verwenden, damit auch URLs angezeigt werden
\bibliography{seminar-lit}
\end{document}
