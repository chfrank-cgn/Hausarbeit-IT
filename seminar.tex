% Angepasste Seminarvorlage
%
\documentclass[12pt,a4paper,fancyheader]{scrartcl}

\usepackage[english]{babel}			% englische Namen/Umlaute
\usepackage[utf8]{inputenc}	    	% Zeichensatzkodierung
\usepackage{fancyhdr}
\usepackage{graphicx}				% Einbinden von Bildern
\usepackage{color}					% Farben wenn es sein muß
\usepackage{hyperref}				% Klickbare Verweise und \autoref{label}
\usepackage{booktabs}				% "Schöne" Tabellen
\usepackage{amsmath}				% Mathematischer Formelsatz AMS
\usepackage{amsfonts}
\renewcommand{\familydefault}{\sfdefault}

%
%	Hier werden Titel, Bearbeiter und das Datum eingetragen
%
\newcommand\svthema{Multi-Cluster Management für Containerumgebungen}
\newcommand\svperson{Christian Frank (\#473088)}
\newcommand\svdatum{\today} % \today für das aktuelle Datum, sonst als Text eintragen
\newcommand\lvname{Wirtschaftsinformatik: IT-Infrastruktur}
\newcommand\lvtyp{WS 2019}
\newcommand\lvinst{FOM - Hochschule für Oekonomie \& Management}
\newcommand\lvbetr{Dipl.-Wirtschaftsinf. (FH) Frank Rudolf Becker}

% Das ist ein 100 Wörter umfassender Blindtext, um das Aussehen von längeren Texten zu testen.
\newcommand{\blindtext}{\textit{Lorem ipsum dolor sit amet, consetetur sadipscing elitr, sed diam nonumy eirmod tempor invidunt ut labore et dolore magna aliquyam erat, sed diam voluptua. At vero eos et accusam et justo duo dolores et ea rebum. Stet clita kasd gubergren, no sea takimata sanctus est Lorem ipsum dolor sit amet. Lorem ipsum dolor sit amet, consetetur sadipscing elitr, sed diam nonumy eirmod tempor invidunt ut labore et dolore magna aliquyam erat, sed diam voluptua. At vero eos et accusam et justo duo dolores et ea rebum. Stet clita kasd gubergren, no sea takimata sanctus est Lorem ipsum dolor sit amet.}}

\hypersetup{ % Thema und Author in die Meta-Daten der PDF
pdftitle={FOM Seminar paper: \svthema} 
pdfauthor={\svperson}
}	
	
\begin{document}

% Hier wird der Titel-Bereich formatiert
% Die zuvor definierten Textbausteine werden hier verwendet.
\title{ \huge\textbf{\svthema} }
\author{ {\svperson} \\ \svdatum }
\date{ \normalsize \centering \includegraphics[width=0.3\textwidth]{FOM}\\ {\lvname} \\ {\lvbetr} \\ {\lvinst} \\ {\lvtyp} }
%eigene Veränderung- FOM Logo mit eingefügt auf dem Deckblatt

\maketitle
\thispagestyle{empty} % lässt die Seitennummer auf der Titelseite verschwinden
\pagenumbering{Roman}

\begin{abstract}
In this paper we'll first have a look at container technologies and orchestration frameworks. After introducing the most popular orchestration framework, Kubernetes, we'll look at challenges for Enterprises to manage multiple Kubernetes clusters.
For a possible solution we'll look at Rancher from Rancher Inc., a solution that offers very interesting features to manage multiple clusters. In closing we'll look at coming developments in container deployments.

\end{abstract}

\cleardoublepage
\tableofcontents			% Inhaltsverzeichnis ... nicht für kurze Dokumente!
\cleardoublepage
\listoffigures				% Abbildungsverzeichnis ... nicht für kurze Dokumente
\cleardoublepage
\pagenumbering{arabic}
\setcounter{page}{4}

%
%	Einfuehrung
%

\pagebreak
\section{Introduction to Cloud Computing and Container Orchestration Frameworks}

\onehalfspacing

\subsection{Enterprise IT}

Ever since the start of the decade, Enterprise IT has undergone a massive transformation towards a utility-like service business. As of 2019, in Enterprise IT, Cloud Computing is now the new norm according to Gartner Research,\footnote{See \textit{Gartner (2019)}: Cloud computing is the new norm. \cite{gartnerCloudStatement}} and is expected to grow even further.\footnote{See \textit{Gartner (2019)}: Gartner Forecasts Worldwide Public Cloud Revenue. \cite{gartnerForecast}}

In the early days, compute transformation was focused on virtualization, whereas cloud computing now focuses on containerization. Both technologies are quite old, virtualization started in the early 70s, pioneered by IBM, with VM/CMS; the first attempt at containerization was made with the implementation of chroot() for Unix System V in the late 70s.

Outside of the mainframe world virtualization technologies were not widespread until VMware commoditized virtualization with ESX/ESXi in the early 00s. Around virtualization, a new ecosystem of self-service portals appeared, such as Microfocus' Cloud Service Automation. Enterprise virtualization with a self-service portal is not cloud computing though - for cloud computing to come to life, the first real open-source cloud operation system, OpenStack was needed, together with the arrival of the major public cloud providers (Amazon Web Services, Microsoft Azure, Google Cloud, Alibaba Cloud).

AWS started initially by providing excess compute capacity to its customers from its internal platforms, before turning into one of the biggest IT providers worldwide.

With all this raw compute power now available on tap, there is no real reason for Enterprise IT anymore to operate in-house data centers.

\subsection{Container Run-time}

During that time, Docker pioneered the first simple orchestration environment to run containers,\footnote{See \textit{Docker (2019)}: Enterprise Container Platform. \cite{docker}} on a single node. Whereas virtualization is focused on virtual compute instances, containerization is focused on application delivery and deployment; with immutable images and a focus on automation, it is fully aimed at the development of cloud-native applications.\footnote{See \textit{Wiggins, A. (2017)}: The Twelve-Factor App. \cite{12factor}}

It is crucial to understand the difference: Whereas virtualization was aimed at the infrastructure level, containerization is aimed at (agile) application development and deployment.

Docker evolved and spawned other container run-timer environments, such as containerd and podman, but they mostly remained focused on executing on a single node. To orchestrate containers on more than one node, an orchestration framework is needed. 

Here are the most popular in 2019:
\begin{itemize}
\item Docker Swarm
\item Kubernetes
\item Mesos DC/OS
\end{itemize}

After Mirantis acquired the Docker Enterprise business, it announced the end of life for Docker Swarm in 2021\footnote{See \textit{Mirantis (2019)}: What We Announced Today and Why it Matters. \cite{mirantisDocker}}; Mesos DC/OS never gathered a large following, so as of the time of writing, CNCF's Kubernetes remains as the only container orchestration framework with a sizable installed base and that is under active development.

In this paper, we'll have a look at one of the challenges posed by introducing Kubernetes as a container orchestration framework into Enterprise IT and showcase a possible solution.


%
%	z.B. Text des zweiten Autors
%

\section{Weitere Kommandos}

\subsection{Mathematische Formeln}

Mathematische Formeln werden mittels \verb|\(...\)|
in den Fließtext eingebaut
--- zum Beispiel \( E=mc^2 \) und:
sei \(V\) ein Vektorraum über \(\mathbb{R}\)
und \(\mathcal{M}\) eine Indexmenge
---
oder aber mittels \verb|\[...\]|
abgesetzt und zentriert dargestellt:
	\[
	\pmb{x} = \sqrt[3]{\frac{a^2-b^2}{a^2+b^2}}
		~~~~~ \text{versus} ~~~~~
	\boldsymbol{x} = \sqrt[3]{\frac{a^2-b^2}{a^2+b^2}}
	\]


\subsection{Silbentrennung}
Vertrauen Sie bitte nie einer automatischen Silbentrennung (auch nicht der von Microsoft Word \& Co.). In folgendem Test-Absatz ist das Wort "`Spracherkennung"' falsch getrennt.

Testzeile Testzeile Testzeile Testzeile Testzeile Testzeile Testzeile Testzeile Te Spracherkennung.

Sie können LaTeX die richtige Trennung mit \verb|\hyphenation{...}| mitteilen. Man tut das üblicherweise noch vor \verb|\begin{document}|.

\hyphenation{Sprach-er-ken-nung}	% hier jetzt zu Demonstrationszwecken im Text
Testzeile Testzeile Testzeile Testzeile Testzeile Testzeile Testzeile Testzeile Te Spracherkennung.


\subsection{Literaturzitate}
\label{sec:literaturzitate}

Im Lehrbuch \cite{Schukat-Talamazzini1995}
finden sich Hinweise auf einschlägige Verfahren der automatischen Spracherkennung.

Das Literaturverwaltungsprogramm JabRef \cite{Kopp2018} ist für viele Plattformen verfügbar und unterstützt bei der Literaturrecherche. Es ist prädestiniert dazu, mit {\LaTeX} in Kombination mit {Bib\TeX} zusammenzuarbeiten.

Über die Literaturrecherche haben Sie Zugriff auf das Buch "`Das Textverarbeitungssystem LaTeX"' \cite{Oechsner2015}. Hierzu können Sie auch direkt den DOI-Link im Literaturverzeichnis anklicken.

\subsection{Einbinden von Bildern}

Sie können mit der Anweisung \verb|\includegraphics{datei}| eine Grafikdatei einbinden. Diese kann im PDF-, JPG- oder PNG-Format vorliegen. Grafiken werden üblicherweise in die float-Umgebung \verb|figure| gekapselt. Der Assistent in TeXstudio \cite{vanderZander2018} tut dies automatisch, wenn das Verhalten nicht explizit abgeschaltet wird. Jede eingebundene Grafik muss vom Text aus referenziert werden. Die textuelle Referenz hat \textit{vor} der Grafik zu erfolgen. Ein eingebundenes PDF ist in \autoref{fig:AufbauMustererkennungssystem} zu sehen.

\begin{figure}[tb]
\centering
\includegraphics [height=20mm] {images/bildchen}
\caption {Aufbau eines Mustererkennungssystems}
\label{fig:AufbauMustererkennungssystem}
\end{figure}

Auch andere Grafikformate werden unterstützt. Verwenden Sie den Assistenten in TeXstudio, um komfortabel Grafiken einzufügen. Siehe dazu \autoref{fig:GrafikEinfuegen}. Die Grafiken finden sich nicht notwendigerweise direkt am Einfüge-Ort. Das Textsatzsystem richtet es so ein, dass es gut aussieht.

\begin{figure}
\centering
\includegraphics[width=0.7\linewidth]{images/GrafikEinfuegen}
\caption{Verwendung des Assistenten, um eine Grafik einzufügen.}
\label{fig:GrafikEinfuegen}
\end{figure}

\blindtext

\blindtext

\blindtext
\blindtext

\subsection{Querverweise}\label{sec:querverweise}
Verwenden Sie unter TeXstudio die rechte Maustaste in der Strukturübersicht, um für einen Abschnitt ein Label zu erzeugen, auf das Sie Bezug nehmen können (siehe \autoref{fig:LabelErzeugen}).

\begin{figure}
\centering
\includegraphics[width=0.4\linewidth]{images/LabelErzeugen}
\caption{Automatische Erzeugung eines Labels für Querverweise}
\label{fig:LabelErzeugen}
\end{figure}

Es wird ein Eintrag \verb|\label{key}| erzeugt, auf den man beispielsweise mit
\verb|\autoref{key}| verweisen kann. Verwendet man \verb|\autoref|, wird der Typ des Objekts
(z.\,B. Abbildung, Tabelle, etc.) mit ausgegeben. Verwendet man nur \verb|\ref|,
so wird nur die Nummerierung des Objekts ausgegeben.

\subsection{Erstellen von Tabellen}

Das Volk hat gesprochen. Siehe \autoref{tab:tabular}. Auch hier kommt ein float zum Einsatz, jedoch mit dem Positionierungs-Hinweis \verb|[h]|. Jedes float sollte mit einem Label versehen werden, und es sollte im Text darauf verwiesen werden, da sich die Position ändern kann.

\begin{table}[h]
\centering
\begin{tabular} {|llc||r|}
	\hline
	Name & Rang & Fraktion & Stimmenanteil \\
	\hline
	Mobutu & General & CDU & 57\% \\
	Tsvangirai & Oberst & CSU & 63\% \\
	\hline
\end{tabular}
\caption {Bundestagswahl in Simbabwe}
\label{tab:tabular}
\end{table}

\autoref{tab:booktab} verwendet keine vertikalen Linien und entspricht dem üblicherweise in Büchern verwenden Stil. Derartige Tabellen sind deutlich ansehnlicher.

\begin{table}[h]
\centering
\begin{tabular} {llcr}
	\toprule
	Name & Rang & Fraktion & Stimmenanteil \\
	\midrule
	Mobutu & General & CDU & 57\% \\
	Tsvangirai & Oberst & CSU & 63\% \\
	\bottomrule
\end{tabular}
\caption {Bundestagswahl in Simbabwe}
\label{tab:booktab}
\end{table}

\subsection{Listen und Aufzählungen}

Listen und Aufzählungen werden in einer Umgebung angelegt (umschlossen von \verb|begin| und \verb|end|.). \verb|\begin{itemize}| leitet eine Liste ein und \verb|\begin{enumerate}| eine Aufzählung. Die Einträge werden jeweils mit \verb|\item| begonnen. Folgend zwei Beispiele.

Itemize:
\begin{itemize}
\item Test
\item Test
\item Test
\end{itemize}

Enumerate:
\begin{enumerate}
\item Test
\item Test
\item Test
\end{enumerate}


% Das Literaturverzeichnis
\cleardoublepage
\bibliographystyle{IEEEtran}	% ieeetran verwenden, damit auch URLs angezeigt werden
\bibliography{seminar-lit}
\end{document}
